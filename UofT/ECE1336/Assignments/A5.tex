\documentclass{article}

\usepackage{amsmath,graphicx,parskip}
\usepackage{fancyhdr}
\pagestyle{fancy}
\lhead{Samuel Huberman}
\chead{ECE1336:A5}
\rhead{999157923}

\newcommand{\unit}[1]{\ensuremath{\, \mathrm{#1}}}
\numberwithin{equation}{section}

\begin{document}


\section*{Problem 1}

\begin{itemize}
\item a. For Maxwell-Boltzmann, the particles are identifiable and do not obey the Pauli exclusion principle. For Fermi-Dirac, the particles are not identifiable and obey the Pauli exclusion principle. For Bose-Einstein, the particles are not identifiable and do not obey the Pauli exclusion principle.

\item b. A gas molecule in equilibrium obeys the Maxwell-Boltzmann distribution. An electron in a free-electron gas obeys the Fermi-Dirac distribution. A photon obeys the Bose-Einstein distribution.

\item c. With 4 objects and 2 boxes, there exists $2^4$ ways to arrange the particles among the boxes without regard for the resulting distribution. The probability of a giving arrangement is the number of independent ways of arriving at that arrangement divided by $2^4$. There are sixteen possible arrangements.
\newline
\begin{tabular}{|l|l|}
\hline
Configuration & Probability\\ \hline
QS1:(A,B,C,D) QS2:(0) & $\frac{1}{2^4}$ \\ \hline
QS1:(A,B,C) QS2:(D) & $\frac{4}{2^4}$ \\ \hline
QS1:(A,B) QS2:(C,D) & $\frac{6}{2^4}$ \\ \hline
QS1:(A) QS2:(B,C,D) & $\frac{4}{2^4}$ \\ \hline
QS1:(0) QS2:(A,B,C,D) & $\frac{1}{2^4}$\\ \hline
QS1:(D) QS2:(A,C,B) & $\frac{4}{2^4}$ \\ \hline
QS1:(C,D) QS2:(A,B) & $\frac{6}{2^4}$ \\ \hline
QS1:(B,C,D) QS2:(A) & $\frac{4}{2^4}$ \\ \hline
QS1:(C) QS2:(A,B,D) & $\frac{4}{2^4}$ \\ \hline
QS1:(C,B) QS2:(A,D) & $\frac{6}{2^4}$ \\ \hline
QS1:(A,C,D) QS2:(B) & $\frac{4}{2^4}$ \\ \hline
QS1:(B) QS2:(A,C,D) & $\frac{4}{2^4}$ \\ \hline
QS1:(B,D) QS2:(A,C) & $\frac{6}{2^4}$ \\ \hline
QS1:(A,B,D) QS2:(C) & $\frac{4}{2^4}$ \\ \hline
QS1:(A,C) QS2:(B,D) & $\frac{6}{2^4}$ \\ \hline
QS1:(A,D) QS2:(B,C) & $\frac{6}{2^4}$ \\ \hline
\end{tabular}
\end{itemize}
\section*{Problem 2}
The number of states, G, between 0 and 1 ev is given by:

\begin{align*}
	G &= \int_0^1 g(\epsilon)d\epsilon
\\       &= \int_0^1 \frac {8\sqrt{2} \pi V}{h^3}m^{\frac{3}{2}}\sqrt{\epsilon}d\epsilon
\\       &= \frac {8\sqrt{2} \pi V}{h^3}m^{3/2}\frac{2\epsilon ^{\frac{3}{2}}}{3}|_0^1
\\       &=4.51190E21
\end{align*}

\section*{Problem 3}
The probability of a state being filled with energy $\epsilon_f+\Delta\epsilon$:
\begin{align*}
	f(\epsilon_f+\Delta\epsilon) &= \frac{1}{1+e^{(\epsilon_f+\Delta\epsilon-\epsilon_f)/kT}}
\\       &= \frac{1}{1+e^{(\Delta\epsilon)/kT}}
\end{align*}
The probability of a state being empty with energy $\epsilon_f-\Delta\epsilon$:
\begin{align*}
	1-f(\epsilon_f-\Delta\epsilon) &=1- \frac{1}{1+e^{(\epsilon_f-\Delta\epsilon-\epsilon_f)/kT}}
\\       &= \frac{1+e^{(-\Delta\epsilon)/kT}-1}{1+e^{(-\Delta\epsilon)/kT}}
\\       &= \frac{1}{1+e^{(\Delta\epsilon)/kT}}
\end{align*}
\end{document}


\documentclass{article}

\usepackage{amsmath,graphicx,parskip}
\usepackage{fancyhdr}
\pagestyle{fancy}
\lhead{Samuel Huberman}
\chead{ECE1336:A12}
\rhead{999157923}

\newcommand{\unit}[1]{\ensuremath{\, \mathrm{#1}}}
\numberwithin{equation}{section}

\begin{document}

\section*{Problem 1}
\begin{itemize}
\item i. Ohmic
\item ii. Rectifying
\item iii. Rectifying
\item iv. Ohmic
\end{itemize}  

\section*{Problem 2}
The divison between are rectifying and ohmic contact occurs when $\Phi_S=\Phi_M$, or equivalently, when the two materials' Fermi levels are identical. Here we assume $\Phi_S=\frac{E_g}{2}+\chi_S$. The difference between the intrinsic Fermi energy and the doped Fermi energy at the transition point:
\begin{align*}
\epsilon_f-\epsilon_{fi}&=\frac{E_g}{2}+\chi_S-\Phi_M\\
			&=0.17[eV]
\end{align*}
Recalling the dependence of Fermi energy upon doping concentration:
\begin{align*}
\epsilon_f-\epsilon_{fi}=\pm kTln\frac{N_d-N_a}{n_i}
\end{align*}
Assuming $T=300K$ and $n_i=1.5E10[cm^{-3}]$, $N_d-N_a=1.076E13[cm^{-3}]$. This is the transition concentration.
\begin{itemize}
\item i. For n-type $N_d-N_a>1.076E13[cm^{-3}]$, the contact is ohmic. For n-type $N_d-N_a<1.076E13[cm^{-3}]$, the contact is rectifying.
\item ii. For p-type $N_d-N_a<1.076E13[cm^{-3}]$, the contact is ohmic. For p-type $N_d-N_a>1.076E13[cm^{-3}]$, the contact is rectifying.
\end{itemize}
\section*{Problem 3}
\begin{itemize}
\item i. Diodes
\item ii. Field-effect transistors 
\end{itemize}
\end{document}

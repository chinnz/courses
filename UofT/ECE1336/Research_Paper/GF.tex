\documentclass{article}

\usepackage{amsmath,graphicx,parskip}
\usepackage{fancyhdr}
\usepackage{amsthm}
\pagestyle{fancy}
\lhead{Samuel Huberman}
\chead{ECE1336: Research Paper}
\rhead{999157923}

\newcommand{\ket}[1]{\left| #1 \right>} % for Dirac bras
\newcommand{\bra}[1]{\left< #1 \right|} % for Dirac kets
\newcommand{\unit}[1]{\ensuremath{\, \mathrm{#1}}}
\newtheorem{mydef}{Thereom}
\numberwithin{equation}{section}
\linespread{2.0}


\begin{document}
\section*{Green's Functions}

Green's functions are defined as solutions of inhomogeneous differential equations of the type ($\tilde{L}$ is a linear differential operator):
\begin{align*}
	\tilde{L}G(\mathbf{r},\mathbf{r'})=\delta(\mathbf{r}-\mathbf{r'})\\
        (z-\tilde{L(\mathbf{r})})G(\mathbf{r},\mathbf{r'};z)=\delta(\mathbf{r}-\mathbf{r'})\\
\tilde{L(\mathbf{r})}\phi_n(\mathbf{r})=\lambda_n\phi_n(\mathbf{r})
\end{align*}
The corresponding matrix definition of a Green's function is:
\begin{align*}
	(A-\lambda B)G&=I\\
        G_{\lambda}&=(A-\lambda B)^{-1}
\end{align*}
\begin{mydef}Fundamental Theorem of Green's Functions
\end{mydef}
Let $G(\mathbf{r},\mathbf{r'})$ be a function which:
Satisfies the differential equation
\begin{align*}
	\tilde{L}G(\mathbf{r},\mathbf{r'})=0
\end{align*}
everywhere in (a,b) except at the point $\mathbf{r}=\mathbf{r'}$.
Satisfies a the given homogeneous boundary conditions.
Is continuous for fixed $\mathbf{r'}$, even at $\mathbf{r}=\mathbf{r'}$.
Has continuous first and second derivative everywhere in (a,b), except at the point $\mathbf{r}=\mathbf{r'}$, where it has a jump discontinuity:
\begin{align*}
	\frac{d}{dx}G(\mathbf{r},\mathbf{r'})|_{\mathbf{r'+}}^{\mathbf{r'+}}=\frac{-1}{p(\mathbf{r'})}
\end{align*}

\begin{align*}
	\delta(\mathbf{r}-\mathbf{r'})L(\mathbf{r})&=\bra{\mathbf{r}}L\ket{\mathbf{r'}}\\
	G(\mathbf{r},\mathbf{r'};z)&=\bra{\mathbf{r}}G(z)\ket{\mathbf{r'}}\\
        \delta(\mathbf{r}-\mathbf{r'})=\bra{\mathbf{r}}\ket{\mathbf{r'}}
\end{align*}

The poles of an appropriate analytic continuation of G in the complex E-plane can be interpreted as the energy (the real pole) and the inverse life-time (the imaginary part).


\end{document}


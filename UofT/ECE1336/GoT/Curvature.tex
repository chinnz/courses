\documentclass{article}

\usepackage{amsmath,graphicx,parskip}
\usepackage{fancyhdr}
\pagestyle{fancy}
\lhead{Samuel Huberman}
\chead{Curvature v Radius}
\rhead{999157923}

\newcommand{\unit}[1]{\ensuremath{\, \mathrm{#1}}}
\numberwithin{equation}{section}

\begin{document}

For a plane curve expressed as $y=f(x)$, the curvature is given as: 
\begin{align*}
\kappa=\frac{|y''|}{(1+y'^2)^{3/2}}
\end{align*}
For any point on a curve, we can define a unique osculating circle, which passes through the point $P$ and a pair of points infinitesimally close to $P$. The curvature of this circle is the same has the curvature at point $P$. The radius of this circle becomes the radius of curvature:
\begin{align*}
\kappa=\frac{1}{R}
\end{align*}
In some applications, like the bending of a beam, the slope $y'$ is assumed to be much smaller than unity, giving:
\begin{align*}
\kappa=|y''|
\end{align*}
The moment relationship for the bending of a beam is, M is the moment, E is the elastic modulus of the material, I is the second moment of inertia about the bending axis, R is the radius of curvature:
\begin{align*}
M=\frac{EI}{R}
\end{align*}
Examining the geometry of the beam:
\begin{align*}
\frac{1}{R}=\frac{d\theta}{ds}
\end{align*}
For small angles:
\begin{align*}
\frac{dy}{dx}=\tan \theta= \theta
\end{align*}
Giving:
\begin{align*}
\frac{1}{R}=\frac{d}{dx}\frac{dy}{dx}
\end{align*}
\end{document}


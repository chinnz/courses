\documentclass{article}

\usepackage{amsmath,graphicx,parskip}
\usepackage{fancyhdr}
\pagestyle{fancy}
\lhead{Samuel Huberman}
\chead{Topic 4}
\rhead{999157923}

\newcommand{\unit}[1]{\ensuremath{\, \mathrm{#1}}}
\numberwithin{equation}{section}

\begin{document}

\textbf{Stress}: A measure of the internal forces experienced by a continous material, in force per unit area. 
\newline
\textbf{Strain}: The change in length per unit length as a result of the stress experienced by the material.
\newline
\textbf{Hooke's Law}: A mathematical description of the linear relationship between force and change in length or relative displacement, related by a constant (spring constant, elastic modulus). 
\newline
\textbf{Elastic Modulus}: The consant relating strain to stress, in the elastic regime of deformation.
\newline
\textbf{Wave Equation}: A PDE describing the equation of motion of a wave travelling through a medium.
\newline
\textbf{Phase Velocity}: The rate of advance of a point of constant phase along the propagation direction of the wave. The speed at which the phase of any one frequency component of the wave travels.
\newline
\textbf{Superposition of sinusoids}: A wave characterized by the original values of $\omega$ and $k$ multiplied by a sinusoidal ``envelope of beats'' of much longer wavelength $\frac{4 \pi}{dk}$.
\newline
\textbf{Group Velocity}: The velocity at which the ``envelope of beats'', the group of waves, are seen to move. The velocity at which the wave transmits energy along the propagation direction.
\newline
\textbf{Dispersion}: A phenomenon arising from the phase velocity's (or group velocity's) dependence upon its frequency.
\newline
\textbf{Displacement of the nth atom}: The deviation from the atom's equilibrium position when a an unbalance force is incurred upon the atom. For example, a vibrational motion results in a periodic motion of the atom about its equilibrium position.
\newline
\textbf{Equation of motion and solutions to the equation}: Through Hooke's law, the atom's displacement is related to the force required to produce the displacement. For a 1-D case, the relative displacement between neighbouring atoms is related by an analogous spring constant to the force. This force balance, when written in the differential form is the equation of motion. The solution to the equation can be found by assuming a plane form description of the displacement of atom $n$, and similarly for atom $n-1$ and $n+1$. 
\newline
\textbf{Dispersion relation}: Upon solving the EOM, we uncover a non-linear relation between the wave vector, propagation constant, $k$ and the freqency. This is formally known as the dispersion relation.
\newline
\textbf{Phase velocity}: Frequency divided by $k$.
\newline
\textbf{Group velocity}: The derivative of frequency with respect to $k$.
\newline
\textbf{Limiting case of $k<<\frac{\pi}{a}$}: The wavelength is much greater than twice the atomic distance and frequency is approximately linear with respect to $k$. In the long wavelength limit, phase and group velocity are equivalent.
\newline
\textbf{Significance of $k=\frac{\pi}{a}$}: The group velocity approaches zero as $k$ approaches 
\newline
\textbf{Normal modes}: The set independent solutions that satisfy the wave equation for a given set of boundary equations.
\newline
\textbf{Atomic displacement and equations of motion (Diatomic)}: The neighbouring atoms have different masses, resulting in different equations of motions from an identical chain of atoms. Here the assumed solutions of neighbouring atoms have different amplitudes.
\newline
\textbf{Dispersion relation (Diatomic)}: Obtained from a quadratic, we find that there exists two frequencies for each $k$.
\newline
\textbf{Acoustic Branch (Diatomic)}: The dispersion relation resulting from the difference between the two terms. In the long-wavelength limit, the vibrations of the atoms are in same direction.
\newline
\textbf{Optical Branch (Diatomic)}: The dispersion relation resulting from the addition of the two terms. In the long-wavelength limit, the vibrations of the atoms are in opposite directions.
\newline
\textbf{Forbidden Frequency Region}: Plotting the dispersion relations, we observe a band where no solution of the wave-equation exists. Attempting to excite vibrations at these frequencies resulting in attenuation or damping byt the lattice. The damping is governed by the frequency, atomic masses and force constant.  
\newline
\textbf{Three-Dimensional Lattices}:
\newline
\textbf{Optical Excitation of Lattice Vibrations in Ionic Crystals}: In ionic crystals, the electric vector of light can excite optical mode vibrations, because the electric field exerts forces on the positive and negative charges in opposite directions. If the incident light frequency is equal to the natural frequency of the lattice, resonance occurs and the vibrations become very large. This gives rise to reradiation of electromagnetic waves at the resonant frequency.
\newline
\end{document}


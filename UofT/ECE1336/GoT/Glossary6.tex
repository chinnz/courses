\documentclass{article}

\usepackage{amsmath,graphicx,parskip}
\usepackage{fancyhdr}
\pagestyle{fancy}
\lhead{Samuel Huberman}
\chead{Topic 6}
\rhead{999157923}

\newcommand{\unit}[1]{\ensuremath{\, \mathrm{#1}}}
\numberwithin{equation}{section}

\begin{document}

\textbf{Classical calculation of Specific Heat}: Atoms vibrate about the equilibrium position, behaving as independent classical harmonic oscillators. The forces as a result of displacement obey Hooke's law. There are no free electrons so the entire heat capacity is due to the excitation of thermal vibrations of the lattice. The average energy of is found by through the ensemble energy by integrating the energy of each oscillator over the Maxwell-Boltzmann distribution. Total internal energy is just the average energy multiplied by the total number of atoms in the systems. The specific heat (at constant volume) is defined as the derivative of the total energy with respect to temperature. 
\newline
\textbf{Einstein Theory of Specific Heat}: Instead of independent classical harmonic oscillators, atoms are assumed to behave as independent quantum harmonic oscillators, thereby restricting the possible energies to discrete values given by (6.2-1). The M-B distribution is still used to describe the statiscal behaviour of the atoms, but the actual quanta of vibrational energy that is absorbed or emitted are not distinguishable and hence B-E statistics are used to describe their behaviour. Since the energy of the oscillators are distributed according to the Boltzmann law, the average energy can be found by the same, but discretized approach as for the classical case (akin to finding a centroid). From the math, we can define an Einstein temperature. The reason for a small specific heat at low temperature can be understood by examining the transfer of energy from a gas to solid. If T is large, such that $kT>>\hbar\omega_0$, then the energy required to excite the solid will occur frequently in gas. Where T is low, a gas molecule will rarely have the required energy to excite the lattice and the crystal is unable to absorb heat from its surroundings.
\newline
\textbf{Debye of Specific Heat}: The atoms no longer vibrate independently of one another. In this picture, it is more effective to work with the normal modes of vibration. The vibrational motion is then described as a superpostion of the normal modes which can be regarded as independent quantum harmonic oscillators where number of normal modes obey M-B statistics. The relation between the internal energy and the normal modes is given by (6.3-1). Using our previous understanding, we work in the wave number space to determine the form of the density of states of the normal modes to calculate the internal energy. Ultimately, we require a expression of $k$ as function of $\omega$, such as the dispersion relation for a monatomic lattic. Debye uses a different form, to avoid the mathematical ugliness of a non-closed integral, by approximating the frequency as linear function of wave number through a proportional constant as seen in equation (6.3-8). This approximation is analogous to the long wavelength condition. Using this approximation, Debye was able to arrive at an expression for internal energy and specific heat.
\newline
\textbf{The Phonon}: The quanta of energy that are absorbed or emitted when the quantum harmonic oscillators transition from one quantum state to another are known as phonons. As a first order approximation, only transitions between adjacent state occur and the quanta of energy due to the transition is expressed in 6.4-1. Phonons obey B-E because they are indistinguishable particles. The Deybe result for specific heat can be obtained through the phonon picture. Beginning with an expression for the number of phonons in the frequency range $d\omega$ about $\omega$ in terms of the density of states of the normal modes, the expression for internal energy can be integrated explicitly. 
\newline
\textbf{Thermal Expansion of Solids}: As a result of the coupling between the normal modes (i.e. not independent harmonic oscillators), anharmonicity arises, manifesting itself in the thermal expansion of solids. In the harmonic case, as the temperature increases, the average interatomic distance would remain the same because the interatomic potential is quadratic and therefore symmetric. In reality, there exists a finite amount of energy that can be applied to separate the lattice, indicating that the harmonic picture is false at large displacements from equilibrium. The asymmetry in the potential shifts the average interatomic distance from zero-temperature case as temperature is increased. Classically, we can show that the coefficient of thermal expansion is independent of temperature, but at low temperatures the classical picture fails. By solving Scrhodinger's equation with an anharmonic potential, we can find the average interatomic distance at low temperatures.
\newline
\textbf{Lattice Thermal Conductivity of Solids}: Here, we only consider the thermal conductivity from the lattice vibrations and neglect the energy carried by free electrons. Phenomenologically speaking, the heat flux in a direction is directly proportional to the temperature gradient in that direction, related by a constant known as the thermal conductivity. Without including anharmonicity, the thermal conductivity of a given lattice will essentially infinite because there exists no way by which the flow of energy is randomized. As such, we consider a gas of phonons. Using the ideal gas analogy, the number of gas particles crossing a infinitesimal area with a velocity $dv$ about $v$ can be determined using a M-B distribution of velocities. The net energy flux is found by relating the energy from each particle to the number of particles crossing the boundary, which gives the thermal conductivity. This procedure is valid for phonons if the average velocity is the speed of sound inside the crystal.
\newline
\end{document}


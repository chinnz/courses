\documentclass{article}

\usepackage{amsmath,graphicx,parskip}
\usepackage{fancyhdr}
\usepackage[english]{babel}
\usepackage{verbatim}
\usepackage[top=3cm,bottom=3cm]{geometry}
\pagestyle{fancy}
\lhead{Samuel Cole Huberman}
\chead{MIE1011: Problem Set 4}
\rhead{999157923}

\newcommand{\ket}[1]{\left| #1 \right>} % for Dirac bras
\newcommand{\bra}[1]{\left< #1 \right|} % for Dirac kets
\newcommand{\unit}[1]{\ensuremath{\, \mathrm{#1}}}
\numberwithin{equation}{section}

\begin{document}

\section*{Question 10}

A. Assuming a linear boundary between vapor and solid, use the Clausius-Clapeyron equation
\begin{align*}
\frac{dP}{dT}&=\frac{s^V-s^S}{v^V-v^S}\\
&=\frac{9.1562-(-1.2210)}{206.132-0.0010908}\\
&=0.0503 [kPa/K]
\end{align*}
Find the y-intercept, b:
\begin{align*}
b&=0.6113-0.0503(273.16)\\
&=-13.128648 [kPa]
\end{align*}
Plug in the temperature, $-6$ C:
\begin{align*}
P&=0.0503(273.15-6)-13.128648\\
&=0.317645 [KPa]
\end{align*}
since $0.4375>0.317645$, the system is in the solid phase.

B. Assuming a linear boundary between liquid and solid, use the Clausius-Clapeyron equation
\begin{align*}
\frac{dP}{dT}&=\frac{s^L-s^S}{v^L-v^S}\\
&=\frac{0-(-1.2210)}{0.001000-0.0010908}\\
&=-13447.137 [kPa/K]
\end{align*}
Find the y-intercept, b:
\begin{align*}
b&=0.6113+13447.137(273.16)\\
&=3.67E6 [kPa]
\end{align*}
Plug in the temperature, $-6$ C:
\begin{align*}
P&=-13447.137(273.15-6)+3.67E6\\
&=77597.35 [KPa]
\end{align*}
An increase of 77597.03 KPa is needed.

C. 
\begin{align*}
Pressure&=\frac{Force}{Area}\\
77597350&=\frac{100*9.81}{0.250(width)}\\
width&=0.00005 [m]
\end{align*}
In making such a calculation, we have assumed the linear dependence of saturation pressure on temperature, which in reality may not be the case.

\section*{Question 11}

A. Solve for T,P:
\begin{align*}
15.16-\frac{3063}{T}&=18.70-\frac{3754}{T}\\
T&=195.198[K]\\
P&=59535.9[Pa]
\end{align*}

B. Latent heat for solid to vapour:
\begin{align*}
q_{VS}=T(s^V-s^S)
\end{align*}
Using
\begin{align*}
\frac{dP}{dT}&=\frac{s^V-s^S}{v^V-v^S}
\end{align*}
We write
\begin{align*}
q_{VS}=(v^V-v^S)T\frac{dP}{dT}=T(s^V-s^S)
\end{align*}
where
\begin{align*}
\frac{dP}{dT}=\frac{3754P_re^{18.7}e^{-3754/T}}{T^2}
\end{align*}
Noting $v^V>>v^L>>v^S$ and $v^V=\frac{RT}{P}$
\begin{align*}
q_{VS}&=v^V\frac{3754P_re^{18.7}e^{-3754/T}}{T}\\
&=\frac{R}{P}3754P_re^{18.7}e^{-3754/T}\\
&=31212.5 [J/mol]
\end{align*}
Latent heat for liquid to vapour:
\begin{align*}
q_{VL}&=v^V\frac{3063P_re^{15.16}e^{-3063/T}}{T}\\
&=\frac{R}{P}3063P_re^{15.16}e^{-3063/T}\\
&=25467.1 [J/mol]
\end{align*}
C. Latent heat for solid to liquid:
\begin{align*}
q_{VS}&=T(s^V-s^S)\\
q_{VL}&=T(s^V-s^L)\\
q_{LS}&=q_{VS}-q_{VL}=T(s^L-s^S)\\
&=31212.5-25467.1=5746.4[J/mol]
\end{align*}
\newpage
\section*{Question 13}
A. The contraints
\begin{align*}
\Delta (V_c+V_R)&=0\\
\Delta (N_{ic})&=0\\
\Delta (U_c+U_R)&=0
\end{align*}
Assume that the solvent's volume does not change (incompressible) as it changes from a pure liquid to a weak solution, we have 
\begin{align*}
\Delta (V_c)&=V_B^i-V_B^f+V_A^i-V_A^f\\
&=V_A^i=V_A^i(T,P)
\end{align*}
%The Gibbs function of the weak solution
%\begin{align*}
%G(T,P,N_1,N_2)=N_1\mu_1^0(T,P)+N_2k_bT\ln\left(\frac{N_2f(T,P)}{N_1}\right)
%\end{align*}
B. Apply the second postulate $\Delta S_c + \Delta S_R \geq 0$ and relying upon the assumption that the change in entropy is due solely to the dissolution of A into B, we write
\begin{align*}
\Delta S_c&=S_B^i-S_B^f+S_A^i-S_A^f\\
&=S_A^i-S_A^f=N_2R\frac{\ln U^{3/2}V_A^i(T,P)}{N^{5/2}_2}
\end{align*}

\end{document}

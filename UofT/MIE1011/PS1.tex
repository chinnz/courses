\documentclass{article}

\usepackage{amsmath,graphicx,parskip}
\usepackage{fancyhdr}
\usepackage[english]{babel}
\usepackage[top=3cm,bottom=3cm]{geometry}
\pagestyle{fancy}
\lhead{Samuel Cole Huberman}
\chead{MIE1011: Problem Set 1}
\rhead{999157923}

\newcommand{\ket}[1]{\left| #1 \right>} % for Dirac bras
\newcommand{\bra}[1]{\left< #1 \right|} % for Dirac kets
\newcommand{\unit}[1]{\ensuremath{\, \mathrm{#1}}}
\numberwithin{equation}{section}

\begin{document}
\large
\section*{Question 1}

A fundamental relation must meet the requirement:
\begin{align*}
S(\lambda U, \lambda V, \lambda N_1,...)&=  \lambda S(U, V, N_1,...)
\end{align*}

For a:
\begin{align*}
S&=C\left (\frac{NU}{V}\right )^{\frac{2}{3}}\\
S(\lambda U, \lambda V, \lambda N)&= C \left( \frac{\lambda^2 NU}{\lambda V} \right )^{\frac{2}{3}}\\
&=C \lambda ^{\frac{2}{3}}\left (\frac{NU}{V}\right )^{\frac{2}{3}}
\end{align*}
$S$ is not a fundamental relation.

For b:
\begin{align*}
S&=C\frac{V^3}{NU}\\
S(\lambda U, \lambda V, \lambda N)&= C\frac{(\lambda V)^3}{\lambda N \lambda U}\\
&=C \lambda \frac{V^3}{NU}
\end{align*}
$S$ is a fundamental relation.

\section*{Question 2}

Given
\begin{align*}
T &= \frac{\partial U}{\partial S}_{V,N_1...N_r}\\
P &= - \frac{\partial U}{\partial V}_{S,N_1...N_r}
\end{align*}
We have
\begin{align*}
\frac{P}{T}&= -\frac{\partial S}{\partial V}_{V,N_1...N_r}\\
&=-\frac{\partial S (U,V,N_1...N_r)}{\partial V}\\ 
\end{align*}
Check that the above relation is a zero-order homogeneous function
\begin{align*}
\frac{\partial S (\lambda U, \lambda V,\lambda N_1... \lambda N_r)}{\partial V}_{\lambda U, \lambda V,\lambda N_1... \lambda N_r}&= \frac{\partial S (\lambda U, \lambda V,\lambda N_1... \lambda N_r)}{\partial \lambda V}\frac{\partial \lambda V}{\partial V}\\
&=\lambda \frac{\partial S (\lambda U, \lambda V,\lambda N_1... \lambda N_r)}{\partial \lambda V}
\end{align*}
Recalling
\begin{align*}
 S (\lambda U, \lambda V,\lambda N_1... \lambda N_r)&= \lambda S (U,V,N_1...N_r)\\
\lambda \frac{\partial S}{\partial V}&=\lambda \frac{\partial S (\lambda U, \lambda V,\lambda N_1... \lambda N_r)}{\partial \lambda V}
\end{align*}
Thus $ \frac{P}{T} $ is an intensive property.

\section*{Question 3}

Begin with the entropy formulation
\begin{align*}
dS&=\frac{1}{T}dU +\frac{P}{V}dV-\sum_{i=1}^r\frac{\mu_i}{T}dN_i\\
&=\frac{\partial S}{\partial U}dU+\frac{\partial S}{\partial V}dV+\frac{\partial S}{\partial N_1}dN_1+\frac{\partial S}{\partial N_2}dN_2
\end{align*}
From the relation of $S$
\begin{align*}
\frac{\partial S}{\partial U}&=\frac{3R(N_1+N_2)}{2U}\\
\frac{\partial S}{\partial V}&=\frac{R(N_1+N_2)}{V}\\
\frac{\partial S}{\partial N_i}&=C+R\ln\frac{VU^{\frac{3}{2}}}{(N_1+N_2)^{\frac{5}{2}}}-R\ln\frac{N_i}{N_1+N_2}-\frac{5}{2}R
\end{align*}
By matching the differential terms, we find
\begin{align*}
\frac{\partial S}{\partial U}&=\frac{1}{T}\\
U&=\frac{3RT(N_1+N_2)}{2}\\
\frac{\partial S}{\partial V}&=\frac{P}{V}\\
P&={R(N_1+N_2)}\\
\frac{\partial S}{\partial N_1}&=-\frac{\mu_1}{T}\\
\mu_1&=-T(C+R\ln\frac{V(\frac{3RT}{2})^{\frac{3}{2}}}{(N_1+N_2)}-R\ln\frac{N_1}{N_1+N_2}-\frac{5}{2}R)
\end{align*}



\end{document}

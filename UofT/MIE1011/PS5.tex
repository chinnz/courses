\documentclass{article}

\usepackage{amsmath,graphicx,parskip}
\usepackage{fancyhdr}
\usepackage[english]{babel}
\usepackage{verbatim}
\usepackage[top=3cm,bottom=3cm]{geometry}
\pagestyle{fancy}
\lhead{Samuel Cole Huberman}
\chead{MIE1011: Problem Set 5}
\rhead{999157923}

\begin{document}

\section*{Question 9}
Given that an ideal gas satisfies
\begin{align*}
\mu (T,P)=\mu (T,P_r) +\bar{R}T\ln \frac{P}{P_r}
\end{align*}
The Gibbs function is
\begin{align*}
G=N(\mu (T,P_r) +\bar{R}T\ln \frac{P}{P_r})
\end{align*}
The specific internal energy is
\begin{align*}
u=Ts-Pv+\mu
\end{align*}
and specific enthalpy is
\begin{align*}
h&=Ts-Pv+\mu+Pv\\
&=Ts+\mu
\end{align*}
Entropy is
\begin{align*}
s&=-\frac{\partial \mu}{\partial T}\\
 &=-\frac{\partial \mu(T,P_r)}{\partial T}-\bar{R}\ln \frac{P}{P_r}
\end{align*}
Substituting back into the equation for enthalpy gives
\begin{align*}
h&=-T\frac{\partial \mu(T,P_r)}{\partial T}-\bar{R}T\ln \frac{P}{P_r}+\mu (T,P_r) +\bar{R}T\ln \frac{P}{P_r}\\
h&=-T\frac{\partial \mu(T,P_r)}{\partial T}+\mu (T,P_r)
\end{align*}
Thus $h$ is only a function of $T$ (but also $P_r$).

\section*{Question 12}

For an isotherm, $T_0$, the pressure as a function of specific volume is
\begin{displaymath}
   P(v) = \left\{
     \begin{array}{lr}
       P_1 + a(v_1 -v) & : v < v_1 (I)\\
       v\frac{P_2-P_1}{v_2-v_1}+ \frac{v_2P_1-v_1P_2}{v_2-v_1}& : v_1 < v < v_2 (II)\\
       P_2 + a(v_2 -v) & :  v_2 < v (III)\\
     \end{array}
   \right.
\end{displaymath}
Restricting $a$ to positive values, the criterion for the areas to be equal is only satisfied if the saturation pressure is the midpoint between $P_1$ and $P_2$ since the areas are defined by similar triangles.
\begin{align*}
P_s=\frac{P_1+P_2}{2}
\end{align*}
Following Table 1.3 in the notes, the stable regions are
\begin{displaymath}
     \begin{array}{lr}
       P_s < P < P_2 & (I)\\
       P_1 < P < P_s & (III)\\
     \end{array}
\end{displaymath}

\end{document}

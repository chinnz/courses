\documentclass{article}

\usepackage{amsmath,graphicx,parskip}
\usepackage{fancyhdr}
\usepackage[english]{babel}
\pagestyle{fancy}
\lhead{Samuel Cole Huberman}
\chead{ECE1448: Problem Set 3a}
\rhead{999157923}

\newcommand{\ket}[1]{\left| #1 \right>} % for Dirac bras
\newcommand{\bra}[1]{\left< #1 \right|} % for Dirac kets
\newcommand{\unit}[1]{\ensuremath{\, \mathrm{#1}}}
\numberwithin{equation}{section}

\begin{document}
\large
\section*{Magnetization as a function of temperature}
From degenerate pertubation theory, the energy correction from the applied magnetic field $B$ is governed by the following equation:
\begin{align*}
 \frac{e\hbar B}{2m_e}[H'][A]=W'[A]
\end{align*} 
Where the elements of $[H']$:
\begin{align*}
	H'_{mn}=\int \psi*_{0m}L_z\psi_{0n}dV
\end{align*}
This yields the following eigenvalue problem:
\[
  \frac{e\hbar B}{2m_e}
 \begin{bmatrix}
  1 & 0 & 0\\
  0 & 0  & 0\\
  0 & 0 & -1 
 \end{bmatrix}
 \begin{bmatrix}
  A_{1} \\
  A_{0} \\
  A_{-1} 
 \end{bmatrix}=W'
 \begin{bmatrix}
  A_{1} \\
  A_{0} \\
  A_{-1} 
 \end{bmatrix}
\]
The eigenvalues are found from solving the cubic equation:
\begin{align*}
	\frac{e\hbar B}{2m_e}(1-W')(0-W')(-1-W')=0
\end{align*}
The splitting of the degeneracy results in three distinct energy states (The lower energy being in direction of $B$):
\begin{align*}
         W'_{1}&=-1.7365E-05 [eV]\\
         W'_0&=  0.0000 [eV]\\
         W'_{-1}&=  1.7365E-05 [eV]
\end{align*}
The Maxwell-Boltzmann distribution has the form:
\begin{align*}
	N_i&=\frac{Nexp(-W_i/kT)}{\sum_iexp(-W_i/kT)}\\
	W_i&=W_0+W'_i 	
\end{align*}
The $exp(W_0/kT)$ will cancel in the numerator and denominator, giving:
\begin{align*}
	N_i&=\frac{Nexp(-W'_i/kT)}{\sum_iexp(-W'_i/kT)}\\
\end{align*}
Substituting in $W'_i={-1.7365E-05,0,1.7365E-05}K$ and repeating the calcution for $T={300,77,4.2}K$:
\\
\\
\begin{table} [h!]
 \begin{center}
  \begin{tabular}{| l |l |l |l|}
  \hline
  $N_i$ & $T=300$ & $T=77$& $T=4.2K$\\
  \hline
   $N_{-1}$& 3.3356&    3.3421&    3.4945\\ \hline
   $N_0$& 3.3333&    3.3333&    3.3308\\ \hline
   $N_{1}$& 3.3311&    3.3246&    3.1747\\ \hline
  \end{tabular}
  \caption{Distribution particles into the three energies (all values are to be multiplied by $1^{23}$}
 \end{center}
\end{table}
\\
Recalling the eigenvalues of $L_z$, the average z-component of angular momentum for a given temperature is:
\begin{align*}
	\bar{\langle L_z \rangle}&=\frac{\sum_i N_i\langle L_{zi} \rangle}{N}\\
				 &=\frac{N_{-1}\times(1)+N_{0}\times(0)+N_{1}\times(-1)}{N}	
\end{align*}
Therefore, the macroscopic z-component of magnetization for a given temperature is:
\begin{align*}
\vec{\mu_z}&=\frac{\bar{\langle L_z \rangle} QN}{2m_e}
\end{align*}
\begin{table} [h!]
 \begin{center}
  \begin{tabular}{| l |l |}
  \hline
  $Temperature$ & $u_z [J/T]$\\
  \hline
   300& 0.0004153\\ \hline
   77& 0.0016180\\ \hline
   4.2& 0.029653\\ \hline
  \end{tabular}
\caption{Z-component of the magnetization}
 \end{center}
\end{table}

\end{document}


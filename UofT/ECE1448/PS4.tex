\documentclass{article}

\usepackage{amsmath,graphicx,parskip}
\usepackage{fancyhdr}
\usepackage[english]{babel}
\usepackage[top=3cm,bottom=3cm]{geometry}
\pagestyle{fancy}
\lhead{Samuel Cole Huberman}
\chead{ECE1448: Problem Set 4}
\rhead{999157923}

\newcommand{\ket}[1]{\left| #1 \right>} % for Dirac bras
\newcommand{\bra}[1]{\left< #1 \right|} % for Dirac kets
\newcommand{\unit}[1]{\ensuremath{\, \mathrm{#1}}}
\numberwithin{equation}{section}

\begin{document}
\large
\section*{Fermi Energies}

To determine the Fermi energy at a given temperature, the following cubic equation must be solved for $\epsilon_f$ where $n,W_1,W_2,W_3,N_1,N_2,N_3$ are known:
\begin{align*}
n=\frac{N_1}{e^{(W_1-\epsilon_f)/kT}+1}+\frac{N_2}{e^{(W_2-\epsilon_f)/kT}+1}+\frac{N_3}{e^{(W_3-\epsilon_f)/kT}+1}
\end{align*}
Since $W_1=0$, the first term on the right hand side approaches $N_1$:
\begin{align*}
n=N_1+\frac{N_2}{e^{(W_2-\epsilon_f)/kT}+1}+\frac{N_3}{e^{(W_3-\epsilon_f)/kT}+1}
\end{align*}
Which is a quadratic equation that can easily be solved:
\begin{align*}
\epsilon_f(174)&=0.8999 [eV]\\
\epsilon_f(290)&=0.8476 [eV]
\end{align*}
Solving the original cubic using MATLAB gives identical results. The number of electrons in the energy level $W_3$:
\begin{align*}
n_3=\frac{N_3}{e^{(W_3-\epsilon_f(T))/kT}+1}
\end{align*}
For $T=174$:
\begin{align*}
n_3&=\frac{2E23}{e^{(1-0.8999)/(8.617E-5\times174)}+1}\\
   &=2.5044E20 [cm^{-3}]
\end{align*}
For $T=290$:
\begin{align*}
n_3&=\frac{2E23}{e^{(1-0.8476)/(8.617E-5\times290)}+1}\\
   &=4.4598E20 [cm^{-3}]
\end{align*}
Vacancies can be found from:
\begin{align*}
p_i=N_i(1-\frac{1}{e^{(W_i-\epsilon_f(T))/kT}+1})
\end{align*}
\begin{table} [!h]
 \begin{center}
  \begin{tabular}{| l |l |l|}
  \hline
  $State$ & $T=174K$& $T=290K$\\
  \hline
   $W_1 [cm^{-3}]$& 0 & 1.7764E8\\ \hline
   $W_2 [cm^{-3}]$& 2.5083E20 &4.4538E20\\ \hline
  \end{tabular}
\caption{Vacancies in $W_1$ and $W_2$}
 \end{center}
\end{table}

\end{document}

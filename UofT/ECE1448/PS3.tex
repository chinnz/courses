\documentclass{article}

\usepackage{amsmath,graphicx,parskip}
\usepackage{fancyhdr}
\usepackage[english]{babel}
\pagestyle{fancy}
\lhead{Samuel Cole Huberman}
\chead{ECE1448: Problem Set 3}
\rhead{999157923}

\newcommand{\ket}[1]{\left| #1 \right>} % for Dirac bras
\newcommand{\bra}[1]{\left< #1 \right|} % for Dirac kets
\newcommand{\unit}[1]{\ensuremath{\, \mathrm{#1}}}
\numberwithin{equation}{section}

\begin{document}
\large
\section*{Z-component of magnetization}
In order to determine the z-component of magnetization, the given wavefunction is expanded in terms of the conventional eigenfunctions of the p-state:
\begin{align*}
 \psi_{11}&=\Sigma A_n \phi_n\\
	  &=f(r)N(A_{11}\frac{(x+iy)}{\sqrt{2}}+A_{10}z+A_{1-1}\frac{(x-iy)}{\sqrt{2}})
\end{align*} 
Or equivalently in bra-ket notation:
\begin{align*}
 \mid\psi_{11}\rangle&=\Sigma A_n \mid \phi_n\rangle\\
		     &=f(r)N(A_{11}\mid\phi_{11}\rangle+A_{10}\mid\phi_{10}\rangle+A_{1-1}\mid\phi_{1-1}\rangle)	
\end{align*}
Rearranging to solve for $A_n$ ($\langle \phi_n \mid=\mid \phi_n\rangle^{\dagger}$):
\begin{align*}
 A_n&=\langle \phi_n \mid \psi_{11}\rangle
\end{align*}
Substituting in the givens (ignoring the factor $f(r)N$ for the moment and including the $\frac{1}{\sqrt{2}}$ in the expansion coefficient where necessary):
\[A_{11}=\frac{1}{\sqrt{2}}
\begin{bmatrix}
1 & -i &0
\end{bmatrix}
\begin{bmatrix}
  1+8i \\
  -4+4i \\
  8+i  
 \end{bmatrix}
=\frac{1}{\sqrt{2}}(5+12i)
\]
\[A_{10}=
\begin{bmatrix}
0 & 0 &1
\end{bmatrix}
\begin{bmatrix}
  1+8i \\
  -4+4i \\
  8+i  
 \end{bmatrix}
=(8+i)
\]
\[A_{1-1}=\frac{1}{\sqrt{2}}
\begin{bmatrix}
1 & i &0
\end{bmatrix}
\begin{bmatrix}
  1+8i \\
  -4+4i \\
  8+i  
 \end{bmatrix}
=\frac{1}{\sqrt{2}}(-3+4i)
\]
Recalling the possible measurements of the $\tilde{L_z}$:
\begin{align*}
\tilde{L_z}\mid\phi_{11}\rangle&=\hbar\mid \phi_{11}\rangle\\
\tilde{L_z}\mid\phi_{10}\rangle&=0\mid \phi_{10}\rangle\\
\tilde{L_z}\mid\phi_{1-1}\rangle&=-\hbar\mid \phi_{1-1}\rangle
\end{align*}
We can now write the expression of the ensemble average of the expectation values of the z-component of angular momentum:
\begin{align*}
\bar{\langle L_z \rangle} &=\frac{\langle\psi_{11}\mid \tilde{L_z}\mid\psi_{11}\rangle}{\langle\psi_{11}\mid\psi_{11} \rangle}\\
		       &=\frac{\mid A_{11}A_{11}^* \mid\langle\phi_{11}\mid \tilde{L_z}\mid\phi_{11}\rangle+\mid A_{10}A_{10}^*\mid \langle\phi_{10}\mid \tilde{L_z}\mid\phi_{10}\rangle+\mid A_{1-1}A_{1-1}^*\mid\langle\phi_{1-1}\mid \tilde{L_z}\mid\phi_{1-1}\rangle}{\mid A_{11}A_{11}^*\mid\langle\phi_{11}\mid\phi_{11}\rangle+\mid A_{10}A_{10}^*\mid\langle\phi_{10}\mid\phi_{10}\rangle+\mid A_{1-1}A_{1-1}^*\mid \langle\phi_{1-1}\mid\phi_{1-1}\rangle}\\
                       &=\frac{169\hbar/2+65(0)-25\hbar/2}{65+25/2+169/2}\\
                       &=\frac{144\hbar}{324}\\
		       &=\frac{4\hbar}{9}
\end{align*}
Therefore, the macroscopic z-component of magnetization is:
\begin{align*}
\vec{\mu_z}&=\frac{\bar{\langle L_z \rangle} QN}{2m_e}\\
     &=\frac{(10^{23})4\hbar e}{18m_e}\\
     &=0.271 [J/T]
\end{align*}

\section*{Justification of quantum number selection}
We perform a transformation of coordinates upon the wavefunction:
\begin{align*}
 \psi_{11}'&=T^{-1}\psi_{11}
\end{align*}
\[
 \begin{bmatrix}
  \psi_{x'} \\
  \psi_{y'}  \\
  \psi_{z'}   
 \end{bmatrix}
  =\frac{f(r)N}{9}
 \begin{bmatrix}
  1 & -4 & 8\\
  8 & 4  & 1\\
 -4 & 7 & 4 
 \end{bmatrix}
 \begin{bmatrix}
  1+8i \\
  -4+4i \\
  8+i  
 \end{bmatrix}
\]
\[
\psi_{11}'=f(r)N
\begin{bmatrix}
  9 \\
  9i  \\
  0   
 \end{bmatrix}
\]
\begin{align*}
\psi_{11}'=9\sqrt{2}f(r)N\frac{(x+iy)}{\sqrt{2}}
\end{align*}
Because of the physical laws do not vary upon the application of a coordinate transformation and hence the physical significance of quantum numbers is invariant, we can say, independent of the frame of reference, the given wavefunction represents the p-state $l=1$ and $m=1$. 
\end{document}
Following the conventional procedure by operating upon ($\tilde{L_z}$) the wavefunction to calculate the measurable observable ($L_z$):
\begin{align*}
 \tilde{L_z'}\psi_{11}'&=L_z'\psi_{11}'\\
	              &=\frac{\hbar}{i}(x'\frac{\partial}{\partial y'}-y'\frac{\partial}{\partial x'})(9x'+9iy')\\
		      &= \hbar(9x'+9iy')f(r)N	
\end{align*}
The magnetic moment in the $z'$ direction is therefore:
\begin{align*}
 \mu_z'&=\frac{\hbar N Q}{2m_e}
\end{align*}
The magnetic moment in the $z$ direction is therefore:
\begin{align*}
 \mu_z&=T\mu_z'\
\end{align*}
\[
=\frac{1}{9}
 \begin{bmatrix}
  1 & 8 & -4\\
  -4 & 4  & 7\\
  8 & 1 & 4 
 \end{bmatrix}
 \begin{bmatrix}
  0 \\
  0  \\
  \frac{\hbar N Q}{2m_e}   
 \end{bmatrix}
\]
\[
\mu_z=\frac{1}{9}
 \frac{\hbar N Q}{2m_e}   
 \begin{bmatrix}
  -4 \\
  7  \\
  4 \\
 \end{bmatrix}
\]
Where $N=10^{23}$.

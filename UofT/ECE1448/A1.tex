\documentclass{article}

\usepackage{amsmath,graphicx,parskip}
\usepackage{fancyhdr}
\pagestyle{fancy}
\lhead{Samuel Cole Huberman}
\chead{ECE1448}
\rhead{999157923}

\newcommand{\unit}[1]{\ensuremath{\, \mathrm{#1}}}
\numberwithin{equation}{section}

\begin{document}

%\title{ECE1448: Quantum Mechanics for Engineers}
%\author{Samuel Huberman (ID: 999157923)}
%\date{2011-09-16}
%\maketitle

\section*{3D Particle in a Box}

Using separation of variables [$\Psi(x,y,z)=X(x)Y(y)Z(z)$] to solve the 3D Schrodinger Equation for a particle in a (cubic) box, we find:
\begin{align*}
	-\frac{\hbar^2}{2m}\frac{X''}{X} &= W_1
\\	-\frac{\hbar^2}{2m}\frac{Y''}{Y} &= W_2
\\	-\frac{\hbar^2}{2m}\frac{Z''}{Z} &= W_3
\end{align*}
Where each equation is equivalent to that describing a 1D particle in the box with the same boundary conditions ($[X(0),Y(0),Z(0)]=0$ and $[X(a),Y(a),Z(a)]=0$) Therefore, we can write:
\begin{align*}
	X(x)_{n}=\sqrt{\frac{1}{2a}}\sin (\frac{n_x \pi}{a}x)
\\	Y(y)_{n}=\sqrt{\frac{1}{2a}}\sin (\frac{n_y \pi}{a}y)
\\	Z(z)_{n}=\sqrt{\frac{1}{2a}}\sin (\frac{n_z \pi}{a}z)
\end{align*}
The overall wavefunction is then:
\begin{align*}
	\Psi(x,y,z)_{n}=\sqrt{\frac{8}{a^3}}\sin (\frac{n_x \pi}{a}x)\sin (\frac{n_y \pi}{a}y)\sin (\frac{n_z \pi}{a}z)
\end{align*}
The energies are:
\begin{align*}
	W_{1n}=\frac{n_x^2\pi^2\hbar^2}{2ma^2}
\\	W_{2n}=\frac{n_y^2\pi^2\hbar^2}{2ma^2}
\\	W_{3n}=\frac{n_z^2\pi^2\hbar^2}{2ma^2}
\end{align*}
We observe what is formally known as degeneracy, where different combinations of of $n_x,n_y,n_z$, bring about the same total energy of the system.
\begin{align*}
	W_{n}=W_{1n}+W_{2n}+W_{3n}
\end{align*}

\begin{center}
\begin{tabular}{| l |c |r |}
  \hline
  $\frac{W}{W_{1n}}$ & $(n_x,n_y,n_z)$ & Degeneracy \\
  \hline
  1 & (1,1,1) & 1 \\ \hline
  3 & (2,1,1),(1,2,1),(1,1,2) & 3 \\ \hline
  9 & (2,2,1),(2,1,2),(1,2,2) & 3 \\ \hline
  11 & (3,1,1),(1,3,1),(1,1,3) & 3 \\ \hline
  12 & (2,2,2) & 1 \\ \hline
\end{tabular}
\end{center}

\end{document}

